\documentclass[twoside,11pt]{article}
\PassOptionsToPackage{hyphens}{url}
\usepackage{jmlr2e}
\usepackage{amsmath}
\usepackage[toc,page]{appendix}
\usepackage[table]{xcolor}
\usepackage[marginparsep=30pt]{geometry}
\usepackage{stmaryrd}
\usepackage{algorithm}
\usepackage{algorithmic}
\usepackage{tikz}
\usepackage{pgfplots}
\usepackage{tabu}
\usepackage{longtable}
\usepackage{tabularx}
\usepackage{listings}
\usepackage{fancyref}
\usepackage{relsize}
\usepackage{float}
\usepackage{subcaption}

\usetikzlibrary{%
    arrows,
    arrows.meta,
    decorations,
    backgrounds,
    positioning,
    fit,
    petri,
    shadows,
    datavisualization.formats.functions,
    calc,
    shapes,
    shapes.multipart,
    matrix,
    plotmarks
}

\usepgfplotslibrary{fillbetween}

\pgfplotsset{
  compat=1.3,
  every non boxed x axis/.style={
  enlarge x limits=false,
  x axis line style={}%-stealth},
  },
  every boxed x axis/.style={},
  every non boxed y axis/.style={
  enlarge y limits=false,
  y axis line style={}%-stealth},
  },
  every boxed y axis/.style={},
}

\def\perc{\texttt{perco\-late}}
\def\v{\texttt{v0.1.0}}

\def\titl{Programming Skills coursework II b: benchmarking
  \perc{} \v{} with different densities}

\title{\titl}

\author{}

\ShortHeadings{B160509}{B160509}
\firstpageno{1}


\begin{document}

\maketitle

\begin{abstract}
\end{abstract}

\begin{keywords}
\end{keywords}

\section{Introduction} % {{{

This paper documents the results of a benchmark performed
on \perc{} \v{}.
\perc{} is a scientific program written in the
Fortran programming language. It generates a random matrix
with two kinds of cells: empty and filled.
Empty cells build clusters with their neighbors and
\perc{} finds clusters that begin at the first column and
end at the last.
If such a cluster exists, the matrix percolates.

One way to configure \perc{} is by setting the density of
the empty cells.
This benchmark looks at the behavior of \perc{} with
different densities of empty cells and how the density
influences the execution time.
It consists of two phases.
During the first phase of the benchmark, the execution time
of \perc{} with different densities is described and
analyzed.
The second phase looks how the execution time splits onto
subroutines of \perc{} with different densities, to see
where the bottlenecks lie concerning the execution speed.

The benchmark was performed on the Cirrus supercomputer
with exclusive access to one back end node
\citep[see][]{cirrus}.
The program was compiled with the GNU Fortran Compiler
(\texttt{gfortran}) version 4.8.5, with the maximum
optimization provided (optimization level \texttt{O3})
\citep[see][]{gfortran}.

This paper begins by describing \perc{} \v{} and the
conducted benchmark.
Afterwards the results are presented and discussed.
At last a conclusion is drawn and next steps for a
follow-up benchmark are outlined.
% }}}

\section{Method} % {{{

Let $n \in \mathbb{N}$ be a positive integer.
Let $A: n \times n$ be a matrix,
$A \in \mathbb{N}_{0}^{n \times n}$.
$A(i, j); 1 \leq i, j \leq n; i, j \in \mathbb{N}$ is
the cell of $A$ in the $i$th row and the $j$th column.
Let $S := \{filled, empty\}$ be a binary set containing the
two states a cell of $A$ can have and let $\sigma:
\mathbb{N}_0 \rightarrow S$ be a function that
maps a cell of $A$ to its state:
\begin{align*}
\sigma(x) = \begin{cases}
  filled &\text{if } x = 0 \\
  empty  &\text{otherwise}
\end{cases}.
\end{align*}
Let $A': n+2 \times n+2$ be a version of $A$ with a halo
containing filled cells:
\begin{align}
  \label{eq:a'}
i,j = 0, \dots, n+1: A'(i, j) := \begin{cases}
  0 &\text{if } i=0 \lor j=0 \lor i=n+1 \lor j=n+1 \\
  A(i, j) &\text{otherwise}
\end{cases}.
\end{align}
The density of the empty cells $\rho$ of $A$ is determined
by the following function:
\begin{align*}
\rho(A) := \frac{|\{i,j=1,\dots,n: \sigma(A(i,j)) = empty\}|}{n^2}.
\end{align*}

\perc{} \v{} randomly initializes $A$ and sets $A'$
according to (\ref{eq:a'}) (see Algorithm~\ref{alg:perc},
lines 1--2).
After initialization, clusters in $A'$ are build
iteratively (see Algorithm~\ref{alg:build_clusters}).
This is done by setting all empty cells of $A'$ to their
biggest neighbor.
Let $\mu:\mathbb{N}_0^{m \times m} \times \mathbb{N} \times
\mathbb{N} \rightarrow \mathbb{N}_0; m \in \mathbb{N}$ be
the function returning the biggest value of all neighboring
cells and the current cell:
\begin{align*}
  \mu(A, i, j) := \max(A(i, j), A(i-1, j), A(i+1, j),
                       A(i, j-1), A(i, j+1)).
\end{align*}
The clustering is finished, once no cell changes value
anymore (see Algorithm~\ref{alg:build_clusters}).

After the clustering the clusters are sorted in descending
order based on their size
($size: \mathbb{N}_0^{n \times n} \times \mathbb{N}
\rightarrow \mathbb{N}_0; size(A, x) :=
|\{i,j=1,\dots,n: A(i,j) = x\}|$)(see
Algorithm~\ref{alg:perc}, line 5).
Sorting is done using the quicksort algorithm
\citep[see][]{hoare_1961}.

Afterwards every empty cell of $A$ is mapped to its index
in the array containing the sorted clusters.
Every filled cell is mapped to $|clusters| + 1$.
The result of this operation is the color map (see
Algorithm~\ref{alg:perc}, line 6).
The color map is needed for writing a Portable Gray Map
Image (PGM).

After doing this operations, \perc{} writes a log to
\texttt{stdout} and writes $A$ and the color map to files.
These io-operations are not measured during the benchmark.

\begin{algorithm}
  \caption{: \perc{}}
  \label{alg:perc}

  \begin{algorithmic}[1]
    \STATE{randomly initialize $A$}
    \STATE{set $A'$ according to (\ref{eq:a'})}
    \STATE{build\_clusters($A'$)}
    \STATE{$A := A'(1:n, 1:n)$}
    \STATE{sort the clusters based on their size}
    \STATE{build the color map}
  \end{algorithmic}
\end{algorithm}

\begin{algorithm}
  \caption{: build\_clusters($A'$)}
  \label{alg:build_clusters}

  \begin{algorithmic}[1]
    \STATE{$A''(i, j) := 0; i,j = 0,\dots,n+1$}
    \WHILE{$\Sigma_{i,j=1}^{n}(A'(i,j) - A''(i, j)) > 0$}
      \STATE{$A'' := A'$}
      \FOR{$i, j=1,\dots,n$}
        \STATE{$A'(i, j) = \begin{cases}
          \mu(A', i, j) &\text{if } \sigma(A'(i,j)) = empty
          \\
          0 &\text{otherwise}
        \end{cases}$}
      \ENDFOR
    \ENDWHILE
  \end{algorithmic}
\end{algorithm}

For the benchmark, $n$ was set to $2000$.
\perc{} offers an interface which controls the
initialization behavior.
One can define a density $\rho_{goal}$ and a seed which is
passed to the pseudo random number generator.
During initialization, $\rho_{goal}$ is approximated.

The benchmark ran \perc{} with $\rho_{goal} :=
0.01,0.02,\dots,0.99$. Every $\rho_{goal}$ was tried with
five seeds, resulting in 495 distinct measures, with an
approximately uniform distribution over $\rho$ generated
during the initialization.

The three operations: clustering, sorting and building the
color map were measured (see Algorithm~\ref{alg:perc},
lines 3ff).
The sum of all three measures is the execution time
(io-operations and initialization excluded).

The execution time is measured in seconds.
OpenMP 4.5's \texttt{omp\_get\_wtime} function was used
for the time measuring \citep[see][Chapter 3.4.1]{openmp}.

% }}}

\section{Results} % {{{

% 1st part fig {{{
\begin{figure}[htbp]
\begin{center}
\noindent
% density -> time_sum {{{
\begin{subfigure}[t]{0.49\textwidth}
\begin{tikzpicture}[scale=1]
  \datavisualization[
    scientific axes=clean,
    visualize as scatter/.list={d},
    d={style={mark=x, mark size=1.5pt}},
    y axis={include value={0}, label={time in seconds $t$}},
    x axis={include value={0,1}, label={$\rho$}},
  ]
  data[headline={x, y}, read from file=data/density_time_sum.csv, set=d]
  ;
\end{tikzpicture}
\caption{Scatter plot mapping $\rho$ to the execution time
  of Algorithm~\ref{alg:perc}.}
\label{fig:p1:a}
\end{subfigure}
% }}}
% kde time_sum {{{
\begin{subfigure}[t]{0.49\textwidth}
\begin{tikzpicture}[scale=1]
  \datavisualization[
    scientific axes=clean,
    visualize as line/.list={d},
    y axis={include value={0}, label={P(Time=$t$)}},
    x axis={include value={0}, label={time in seconds $t$}},
  ]
  data[headline={x, y}, read from file=data/kde_time_sum.csv, set=d]
  ;
\end{tikzpicture}
\caption{Gaussian kernel density estimation of the
  execution time. Clearly shows that
  Algorithm~\ref{alg:perc} favors extreme
  times (the function has its maxima at the edges). }
\label{fig:p1:b}
\end{subfigure}
% }}}

% density -> num_iter {{{
\begin{subfigure}[t]{0.49\textwidth}
\begin{tikzpicture}[scale=1]
  \datavisualization[
    scientific axes=clean,
    visualize as scatter/.list={d},
    d={style={mark=x, mark size=1.5pt}},
    y axis={include value={0}, label={amount of iterations $\#i$}},
    x axis={include value={0,1}, label={$\rho$}},
  ]
  data[headline={x, y}, read from file=data/density_num_iter.csv, set=d]
  ;
\end{tikzpicture}
\caption{Scatter plot mapping $\rho$ to the amount of
  iterations of Algorithm~\ref{alg:build_clusters}.}
\label{fig:p1:c}
\end{subfigure}
% }}}
% kde num_iter {{{
\begin{subfigure}[t]{0.49\textwidth}
\begin{tikzpicture}[scale=1]
  \datavisualization[
    scientific axes=clean,
    visualize as line/.list={d},
    y axis={include value={0}, label={P(Iter=$\#i$)}},
    x axis={include value={0}, label={amount of iterations $\#i$}},
  ]
  data[headline={x, y}, read from file=data/kde_num_iter.csv, set=d]
  ;
\end{tikzpicture}
\caption{Gaussian kernel density estimation of the
  amount of iterations. Clearly shows that
  Algorithm~\ref{alg:build_clusters} favors extreme amounts
  of iterations (the function has its maxima at the edges).
}
\label{fig:p1:d}
\end{subfigure}
% }}}

% time_sum -> num_iter {{{
\begin{subfigure}[b]{0.49\textwidth}
\begin{tikzpicture}[scale=1]
  \datavisualization[
    scientific axes=clean,
    visualize as scatter/.list={d},
    visualize as line/.list={d1},
    d={style={mark=x, mark size=1.5pt}},
    d1={style={visualizer color=black!50},
        label in legend={text={linear regression line} }},
    y axis={include value={0}, label={amount of iterations $\#i$}},
    x axis={include value={0}, label={time in seconds $t$}},
    %legend=below,
  ]
  data[headline={x, y}, read from file=data/time_sum_num_iter.csv, set=d]
  data[headline={x, y}, read from file=data/lin_reg_time_sum_num_iter.csv, set=d1]
  ;
\end{tikzpicture}
\caption{Plot showing the correlation of the execution time
of Algorithm~\ref{alg:perc} and the amounts of iterations
of Algorithm~\ref{alg:build_clusters}.}
\label{fig:p1:e}
\end{subfigure}
% }}}

\caption{Results of the first phase of the benchmark.}
\label{fig:p1}
\end{center}
\end{figure}
% }}}

This benchmark consists of two phases.
The first phase measures the overall execution time (the
sum of the execution time of the three operations, see
previous chapter). The density $\rho$ of $A$ is mapped to
the relating execution time $t$ (measured in seconds).
The second phase looks how the three different operations
influence $t$, to see how they scale and to identify the
bottlenecks.

The results of the first phase of the benchmark can be seen
in Figure~\ref{fig:p1}.
The relationship between $\rho$ and the execution time
is clearly not linear---like expected---and rather extreme
(see Figure~\ref{fig:p1:a}).
The program produces a discontinuity like jump at $\rho
\approx 0.58$.
While the overall average $t$ is approximately
$278.49$ seconds, the average $t$ over all measures where
$\rho \leq 0.58$ is just approximately $14.79$ seconds,
while the average $t$ over all measures where $\rho > 0.58$
is approximately $639.33$ seconds.
The average of the measurements where $\rho \leq 0.58$ is
a staggering 2 percent of the average of the measurements
where $\rho > 0.58$.
The density of the execution time is a third factor that
makes the measurement extreme (see Figure~\ref{fig:p1:b}).
Added to this jumping behavior and the scale, the
distribution of the execution time also favors extreme
timings.
The upper and lower quartiles of the distribution of $t$
contain 50 percent of all measurements.
This means, not only is there an extreme jump from very
low execution time to very high, there are also just a
few time measurements that are close to the overall average
execution time of approximately $278.49$ seconds.

Figures~\ref{fig:p1:c} and \ref{fig:p1:d} show the same
relation, $t$ replaced with the amount of iterations $\#i$
of the while-loop in Algorithm~\ref{alg:build_clusters}.
The plots look rather similar to each other.
Figure~\ref{fig:p1:e} shows the relation between $t$ and
$\#i$ and it is clearly linear.
The Pearson correlation coefficient is approximately
$0.999$, very close to a perfect linear relationship,
making it save to say, that $\#i$ is the main cause for
$t$.
The relation is not perfectly linear, because $t$ drops
slightly with higher $\rho$ ($\rho > 0.58$), while $\#i$
stays constant for $\rho > 0.58$ (see
Figures~\ref{fig:p1:a}--\ref{fig:p1:d}).

% 2nd part fig {{{
\begin{figure}
\begin{center}
\begin{tikzpicture}[scale=0.75] % {{{
\begin{axis}[
  tick align=outside,
  axis x line=bottom,
  axis y line=left,
  axis line shift=10pt,
  xlabel=$\rho$,
  ylabel=percentage of $t$,
  label shift=10pt,
  legend style={
    at={(2.5,0.5)},
    anchor=east,
    cells={anchor=west},
    draw=white
  },
]
  \addplot[name path=map, color=black!30] table[header=false, col sep=comma] {data/map_percentage.csv};
  \addlegendentry{build\_clusters($A'$)}
  \addplot[name path=sort, color=black!60] table[header=false, col sep=comma] {data/sort_percentage.csv};
  \addlegendentry{sort the clusters}
  \addplot[name path=color, color=black!90] table[header=false, col sep=comma] {data/color_percentage.csv};
  \addlegendentry{build color map}

  \path[name path=axis] (axis cs:0.009,0) -- (axis cs:0.99,0);

  \addplot[fill=black!30] fill between[of=map and sort];
  \addplot[fill=black!60] fill between[of=sort and color];
  \addplot[fill=black!90] fill between[of=color and axis];
\end{axis}
\end{tikzpicture}
% }}}
\vspace{0.5cm}
\caption{Percentage of $t$ spent for every measured
  operation in Algorithm~\ref{alg:perc}.}
\label{fig:p2}
\end{center}
\end{figure}
% }}}

During the second phase of the benchmark the three
different operations were measured distinctly to see how
they scale with different $\rho$.
The second phase underlines the fact that
Algorithm~\ref{alg:build_clusters} is the main cause of
$t$, especially for the extreme high timings
(for $\rho > 0.58$), where approximately 99 percent of
the execution time was spent for the clustering (see
Figure~\ref{fig:p2}).

For the lower densities---which produce very low
$t$---the clustering takes approximately linearly less with
lower $\rho$.
At its lowest point, the clustering only takes
approximately 45 percent of $t$, while the rest of the time
is spent sorting the clusters.

The building of the color map is constant over $\rho$.
The average time spent on building the clusters is 0.05
seconds and does not deviate much with a standard deviation
of 0.04 seconds.
The sorting takes slightly more time than the building of
the color map, but it is also constant over $\rho$ and
takes only 0.71 seconds on average with a standard
deviation of 0.12 seconds.
Both
Both operations can be deemed irrelevant to $t$ concerning
its scale.
Sorting only plays a role for very low $\rho$ which also
produce very low $t$.

% }}}

\section{Discussion} % {{{

% crazy jumping behavior

% bottleneck is clustering

% other cause than just iterations (cluster size?)

% }}}

\section{Conclusion} % {{{

% find better way of clustering

% find reason for jump

% follow up benchmark -> look at cluster size

% }}}

\bibliography{pscw.bib}

\end{document}
